%%%%%%%%%%%%%%%%%%%%%%%%%%%%%%%%%%%%%%%%%
% Compact Academic CV
% LaTeX Template
% Version 2.0 (6/7/2019)
%
% This template originates from:
% https://www.LaTeXTemplates.com
%
% Authors:
% Dario Taraborelli (http://nitens.org/taraborelli/home)
% Vel (vel@LaTeXTemplates.com)
%
% License:
% CC BY-NC-SA 3.0 (http://creativecommons.org/licenses/by-nc-sa/3.0/)
%
%%%%%%%%%%%%%%%%%%%%%%%%%%%%%%%%%%%%%%%%%

%----------------------------------------------------------------------------------------
%	PACKAGES AND OTHER DOCUMENT CONFIGURATIONS
%----------------------------------------------------------------------------------------

\documentclass[11pt]{article} % Default document font size

\input{structure.tex} % Include the file specifying the document structure and styling

\leftskip=2em
\parindent=-1em

% Set PDF meta-information
\hypersetup{
	pdftitle={Davide Pradovera - Curriculum vit\ae},
	pdfauthor={Davide Pradovera}
}

%----------------------------------------------------------------------------------------
\pagenumbering{gobble}

\begin{document}

%----------------------------------------------------------------------------------------
%	CONTACT AND GENERAL INFORMATION
%----------------------------------------------------------------------------------------

\begin{minipage}[c]{.75\textwidth}
{\huge\bfseries Davide Pradovera} % Name
\bigskip\bigskip\medskip % Whitespace

Office 09.128, University of Vienna \\
Oskar-Morgenstern-Platz 1 \\
1090 Vienna, Austria
\medskip % Whitespace

Mobile: +41 077 95 88 993 % Mobile number
\medskip % Whitespace

Emails: \href{mailto:davide.pradovera@univie.ac.at}{davide.pradovera@univie.ac.at}\\ % Email address
\phantom{Emails: }\href{mailto:davidepradovera@gmail.com}{davidepradovera@gmail.com}\\ % Email address
\textsc{URLs}: \href{https://pradovera.github.io}{https://pradovera.github.io}\\ % Academic/personal website
\phantom{\textsc{URLs}: }\href{https://orcid.org/0000-0003-0398-1580}{https://orcid.org/0000-0003-0398-1580}\\ % Academic/personal website
\end{minipage}\hfill%
\begin{minipage}[c]{.225\textwidth}
\fbox{\includegraphics[width=4cm]{../images/profile}}
\end{minipage}\hfill%

\smallskip % Whitespace between contact information and specific CV information

%------------------------------------------------

Born: October 9, 1993---Piacenza, Italy. % Date of birth

Nationality: Italian. % Nationality

%------------------------------------------------

\medskip % Whitespace between contact information and specific CV information

\section*{Current position}

\emph{University assistant and post-doctoral researcher}, Chair of Numerics of PDEs, University of Vienna. % Current or most recent employment position

%------------------------------------------------

\section*{Areas of specialization}

Numerical mathematics for partial differential equations, approximation theory, model order reduction, frequency-domain applications, scattering problems.

%----------------------------------------------------------------------------------------
%	WORK EXPERIENCE
%----------------------------------------------------------------------------------------

\section*{Appointments held}

\years{2014--2017}\hspace{\parindent}\emph{Special courses teacher}, Piacenza (I).

\years{2016}\emph{Developer intern}, Iren S.p.A., Piacenza (I).

\years{2017--2021}\emph{Doctoral assistant}, EPFL, Lausanne (CH).

\years{2022}\emph{Post-doctoral researcher}, EPFL, Lausanne (CH).

\years{2022--now}\emph{University assistant and post-doctoral researcher}, University of Vienna, Vienna (A).

%----------------------------------------------------------------------------------------
%	EDUCATION
%----------------------------------------------------------------------------------------

\section*{Education}

\years{2012--2015}\hspace{\parindent}\emph{\textsc{B.Sc.} in Applied Mathematics} (\emph{cum laude}), Politecnico di Milano, Milan (I).\\
Thesis: ``A mathematical justification of the momentum operator in quantum mechanics''.\\
\phantom{m}\hfill Advisor: Prof. M. Verri.

\years{2015--2017}\emph{\textsc{M.Sc.} in Computational Science and Engineering}, EPFL, Lausanne (CH).\\
Project: ``Implementation of smooth contact mechanics with the mortar method''.\\
\phantom{m}\hfill Advisor: Prof. G. Anciaux.\\
Project: ``Finite elements-based Pad\'e approximants for Helmholtz frequency response problems''.\hspace{9em}\phantom{m} \hfill Advisor: Prof. F. Nobile.\\
Thesis: ``Randomized low-rank approximation of matrices and tensors''.\\
\phantom{m}\hfill Advisor: Prof. D. Kressner.

\years{2017--2021}\emph{\textsc{Ph.D.} in Mathematics}, EPFL, Lausanne (CH).\\
Thesis: ``Model order reduction based on functional rational approximants for parametric PDEs with meromorphic structure''.\hspace{9em}\phantom{m} \hfill Advisor: Prof. F. Nobile.

%----------------------------------------------------------------------------------------
%	GRANTS, HONOURS AND AWARDS
%----------------------------------------------------------------------------------------

\section*{Grants, honors, and awards}

\years{2011}\hspace{\parindent}3\textsuperscript{rd} place at the ``Hong Kong International Science Fair''.

\years{2013}4\textsuperscript{th} place in the ``Championnat International des Jeux Math\'ematiques et Logiques''.

\years{2014}5\textsuperscript{th} place in the ``Championnat International des Jeux Math\'ematiques et Logiques''.

\years{2017}Douchet prize for best GPA, MATH-EPFL.

\years{2020}Prize for exceptional teaching service, Section of Mathematics, EPFL.

\years{2021}Junior Research Fellowship at ESI Vienna.

%----------------------------------------------------------------------------------------
%	PUBLICATIONS AND TALKS
%----------------------------------------------------------------------------------------

\section*{Publications and talks}

\subsection*{Journal articles}

\years{2019}\hspace{\parindent}F. Bonizzoni and DP, ``Distributed sampling for rational approximation of the acoustic scattering of an airfoil'', PAMM 19.

\yearsplus{2020}F. Bonizzoni, F. Nobile, I. Perugia, and DP, ``Fast Least-Squares Pad\'e approximation of problems with normal operators and meromorphic structure'', Math. Comput. 89.

F. Bonizzoni, F. Nobile, I. Perugia, and DP, ``Least-Squares Pad\'e approximation of parametric and stochastic Helmholtz maps'', Adv. Comput. Math. 46.

\yearsminus DP, ``Interpolatory minimal rational model order reduction of parametric problems lacking uniform inf-sup stability'', SIAM J. Numer. Anal. 58.

\yearsplus{2021}F. Bonizzoni and DP, ``Shape optimization for a noise reduction problem by non-intrusive parametric reduced modeling'', Proc. WCCM-ECCOMAS2020.

DP and F. Nobile, ``Frequency-domain non-intrusive greedy Model Order Reduction based on minimal rational approximation'', Sci. Comput. Electr. Eng. 36.

\yearsminus F. Nobile and DP, ``Non-intrusive double-greedy parametric model reduction by interpolation of frequency-domain rational surrogates'', ESAIM:M2AN 55.

\years{2022}DP and F. Nobile, ``A technique for non-intrusive greedy piecewise-rational model reduction of frequency response problems over wide frequency bands'', J. Math. Ind. 12.

\subsection*{Pending articles}

\years{2021}\hspace{\parindent}F. Bonizzoni, DP, and M. Ruggeri, ``Rational-based model order reduction of Helmholtz frequency response problems with adaptive finite element snapshots'', under review.

\subsection*{Presentations at conferences}

\yearsplus{2019}\hspace{\parindent}DP, F. Nobile, F. Bonizzoni, and I. Perugia, ``A technique for rational model order reduction of parametric problems lacking uniform inf-sup stability'', GAMM 2019, Vienna (A).

DP, F. Nobile, F. Bonizzoni, and I. Perugia, ``A technique for rational model order reduction of parametric problems lacking uniform inf-sup stability'', ICIAM 2019, Valencia (E).

\yearsminus DP and F. Nobile, ``Interpolatory rational model order reduction of parametric problems lacking uniform inf-sup stability'', \mbox{ENUMATH} 2019, Egmond aan Zee (NL).

\yearsplus{2021}DP, F. Nobile, and F. Bonizzoni, ``Non-intrusive model reduction of parametric frequency response problems via minimal rational interpolation'', \mbox{ICOSAHOM} 2020/2021 (virtual), Vienna (A).

\yearsminus DP and F. Nobile, ``Non-intrusive model reduction of parametric frequency-response problems -- with applications to UQ'', SIMAI 2020+2021, Parma (I).

\subsection*{Posters}

\yearsplus{2018}\hspace{\parindent}F. Bonizzoni, I. Perugia, F. Nobile, and DP, ``An efficient algorithm for Pad\'e-type approximation of the frequency response for the Helmholtz problem'', \mbox{MoRePaS} IV, Nantes (F).

\yearsminus F. Bonizzoni, I. Perugia, F. Nobile, and DP, ``An efficient algorithm for Pad\'e-type approximation of the frequency response for the Helmholtz problem'', Swiss Numerics Day 2018, Zurich (CH).

\yearsplus{2020}DP and F. Nobile, ``Frequency-domain non-intrusive greedy Model Order Reduction based on minimal rational approximation'', SCEE 2020, Eindhoven (NL).

\yearsminus DP and F. Nobile, ``Frequency-domain non-intrusive greedy Model Order Reduction based on minimal rational approximation'', MORSS 2020 (virtual), Lausanne (CH).

\subsection*{Others}

\yearsplus{2018}\hspace{\parindent}DP, F. Nobile, F. Bonizzoni, and I. Perugia, ``Fast Least-Squares Pad\'e approximation of self-adjoint problems with meromorphic structure'', seminar, \mbox{MATHICSE} retreat, Sainte-Croix (CH).

\yearsminus DP, F. Nobile, F. Bonizzoni, and I. Perugia, ``Fast Least-Squares Pad\'e approximation of self-adjoint problems with meromorphic structure'', workshop talk, DRWA, Alba di Canazei (I).

\years{2019}DP and F. Nobile, ``Polynomial approximation of resonance manifolds'', short seminar, \mbox{MATHICSE} retreat, Champ\'ery (CH).

\yearsplus{2020}DP, ``Pad\'e approximation: a quick overview'', seminar (virtual), CSQI talks, Lausanne (CH).

DP, ``From Pad\'e approximation to rational interpolation'', seminar (virtual), CSQI talks, Lausanne (CH).

DP, ``Minimal rational approximation'', seminar (virtual), CSQI talks, Lausanne (CH).

\yearsminus DP, ``Minimal rational approximation: a model reduction tool for parametrized PDEs with resonances'', seminar (virtual), PDE Afternoons, Vienna (A).

\years{2021}DP, ``Matching-based pMOR for dynamical systems'', seminar (virtual), CSQI talks, Lausanne (CH).


%----------------------------------------------------------------------------------------
%	TEACHING
%----------------------------------------------------------------------------------------

\section*{Teaching experience}

\years{2017}\hspace{\parindent}Analyse avanc\'ee I, Mathematics, EPFL.

\yearsplus{2018}Analyse numerique, Mechanical Engineering, EPFL.

\yearsminus Analyse fonctionnelle, Mathematics, EPFL.

\years{2019}Introduction to partial differential equations, Mathematics, EPFL.

\years{2021}Numerical analysis and computational mathematics, Computational Sciences, EPFL.

\years{2019--2021}Parallel and high-performance computing, Computational Sciences, EPFL.

\smallskip

\hspace{-\leftskip}(Including preparation of course\&exercise material, preparation and grading of assignments\&exams.)

\section*{Other service}

\years{2019}\hspace{\parindent}Supervision of B.Sc. thesis: ``Approximation num\'erique du spectre des op\'erateurs elliptiques d'ordre deux'' by T. Chanay, EPFL.

\yearsplus{2020}Conference organizer, Model Order Reduction Summer School 2020 (virtual event).

\yearsminus Referee for scientific journals: Advances in Computational Mathematics.

\years{2022}Supervision of M.Sc. project: ``Rational approximation for the frequency response of the time-harmonic Maxwell's equations'' by F. Matti, EPFL.

\section*{Computer skills}

\years{Advanced}\hspace{\parindent}Matlab, C/C++, OpenMP, MPI, Python, FreeFem++, \LaTeX.

\years{Intermediate}CUDA, C\#, HTML.

\years{Basic}R, OpenFOAM, Fluent, Fortran, Java.

%----------------------------------------------------------------------------------------
%	LANGUAGES SECTION
%----------------------------------------------------------------------------------------

\section*{Languages}

\begin{minipage}[t]{.45\textwidth}
\begin{tabular}{rl}
Italian: & Mothertongue\\
French: & Intermediate\\
German: & Basic\\
\end{tabular}
\end{minipage}%
\begin{minipage}[t]{.45\textwidth}
\begin{tabular}{rl}
English: & Fluent\\
Japanese: & Basic\\
Spanish: & Basic\\
\end{tabular}
\end{minipage}

%----------------------------------------------------------------------------------------
%	FINAL FOOTER
%----------------------------------------------------------------------------------------

% Any final footer text such as a URL to the latest version of this CV, last updated date, compiled in XeTeX, etc
\begin{center}
	\scriptsize
	\raisebox{-0.5pt}{\textbullet}~~Last updated: \today~~\raisebox{-0.5pt}{\textbullet}~~Vienna~~\raisebox{-0.5pt}{\textbullet}
\end{center}

\end{document}
